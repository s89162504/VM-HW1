\documentclass[12pt]{article}
\usepackage{enumerate, setspace, float, stackengine,}
\usepackage[margin=1.5cm]{geometry}
\usepackage{caption, subcaption}
\usepackage[procnames]{listings}
\usepackage{lstautogobble, color}
\usepackage{amsmath, amsfonts, amssymb}
\usepackage{fontspec}   %加這個就可以設定字體
\usepackage{xeCJK}       %讓中英文字體分開設置
\setCJKmainfont{標楷體} %設定中文為系統上的字型,而英文不去更動,使用原TeX字型

%----------------------------------------------------------------------------------------
%   LSTSETTING
%----------------------------------------------------------------------------------------
\lstdefinestyle{Structure}
{
    language=C,
    morekeywords={target_ulong},
    keywordstyle=\color{blue}, 
    commentstyle=\color{red},
    basicstyle=\ttfamily,
    numbers=none,
    frame=single,
    breaklines=true,
    tabsize=4,
    escapeinside={(*}{*)},
    autogobble=true
}

\lstdefinestyle{Function}
{
    language=C,
    morekeywords={target_ulong, CPUState, shadow_pair, TCGv_ptr, TCGv, TranslationBlock},
    keywordstyle=\textbf, 
    basicstyle=\ttfamily,
    numbers=none,
    frame=single,
    breaklines=true,
    tabsize=4,
    escapeinside={(*}{*)},
    autogobble=true
}

\title{\textbf{Virtual Machine - Assignment 1}}

\author{R04922058 鄭以琳}
\date{\large \today}


\begin{document}
\maketitle

\section{Shadow Stack}
    \subsection{Data Structure}
        \begin{lstlisting}[style=Structure]
            //cpu-defs.h
            void *shack;
            void *shack_top;
            void *shack_end;
            struct hash_list *shadow_hash_list;
        \end{lstlisting}
        
        Variables \verb|shack|, \verb|shacks_top| and \verb|shack_end| are the same as which defined in homework guidlines.
        However, I modified \verb|hash_list|.
        It now contains several buckets, each pointing to the head of a \verb|shadow_pair| chain.
        \\
        \begin{lstlisting}[style=Structure]
            //optimization.h
            struct shadow_pair{
                target_ulong guest_eip;
                unsigned long *host_eip;
                struct shadow_pair *nxt;
            };

            struct hash_list{
                struct shadow_pair *head[SHACK_BUCKET];
            };
        \end{lstlisting}
        
        In a \verb|shadow_pair|, \verb|guest_eip| represents the guest return address, and \verb|host_eip| represents the host return address.
        \verb|nxt| is a pointer points to the next \verb|shadow_pair| when hash collision occurs.
        
        \verb|hash_list| contains $2^{16}$ entries (defined as \verb|SHACK_BUCKET| in \verb|optimization.c|).
        \verb|head| holds the pointers of the first elements of the entries.

        When function \verb|push_shack()| is called, the address of the corresponding \verb|shadow_pair| will be pushed onto the stack. 
        A \verb|shadow_pair| will be updated if \verb|shack_set_shadow()| is called. The function will use \verb|shadow_hash_list| to get the corresponding \verb|shadow_pair| and update the host return address.

    \newpage

    \subsection{Functions}
        % shack_init
        \begin{lstlisting}[style=Function]
            void (*\textbf{\textcolor{blue}{shack\_init}}*)(CPUState *env)
        \end{lstlisting}
        In this funcion, shadow stack and hash list are initialized.
        \\\\
        % find_hash_pair
        \begin{lstlisting}[style=Function] 
            shadow_pair* (*\textbf{\textcolor{blue}{find\_hash\_pair}}*)(CPUState *env, target_ulong next_eip)
        \end{lstlisting}
        This is a function that helps finding the conrresponding \verb|shadow_pair| with a given guest return address.
        If no corresponding \verb|shadow_pair| is found, it will create a new one and return the address.

        There are $2^{16}$ entries in the hash list.
        Thus the 16 least significant bits of \verb|next_eip| are used as hash key.

        When a hash collision occurs, the new \verb|shadow_pair| is added to the head of the chain of the corresponding entry,
        and \verb|hash_list->head| is modified to point to the new \verb|shadow_pair|.
        \\\\
        % shack_set_shadow
        \begin{lstlisting}[style=Function] 
            void (*\textbf{\textcolor{blue}{shack\_set\_shadow}}*)(CPUState *env, target_ulong guest_eip, unsigned long *host_eip)
        \end{lstlisting}
        When a new trasnlation block is created, this function will be called.
        It uses \verb|find_hash_pair()| to retrieve the \verb|shadow_pair| corresponding to \verb|guest_eip|, 
        and update \verb|host_eip| in the \verb|shadow_pair|.
        \\\\
        % helper_shack_flush
        \begin{lstlisting}[style=Function] 
            void (*\textbf{\textcolor{blue}{helper\_shack\_flush}}*)(CPUState *env)
        \end{lstlisting}
        This function flushes the entire shadow stack by setting \verb|shack_top| to \verb|shack|.
        \\\\
        % push_shack
        \begin{lstlisting}[style=Function] 
            void (*\textbf{\textcolor{blue}{push\_shack}}*)(CPUState *env, TCGv_ptr cpu_env, target_ulong next_eip)
        \end{lstlisting}
        The push operation contains serveral steps.
        \begin{enumerate}
            \item Get the corresponding \verb|shadow_pair|. \\
                Call \verb|find_hash_pair()| to retrieve the \verb|shadow_pair| that contains \verb|next_eip|.
                Since the address of the corresponding \verb|shadow_pair| will not change,
                this step does not need to be written in TCG.
                The address retrieved at translation time will be used to generate TCG codes in the following steps.

            \item Check if need to flush the stack. \\
                If \verb|shack_top| equals to \verb|shack_end|, the stack is full and needs to be flushed.
                Flushing simply sets \verb|shack_top| to \verb|shack|.

            \item Push the address of \verb|shadow_piar| onto the stack. \\
                The address of \verb|shadow_pair| retrieved in the first step will be pushed onto the statck.
        \end{enumerate}

        \newpage
        
        % pop_shack
        \begin{lstlisting}[style=Function] 
            void (*\textbf{\textcolor{blue}{pop\_shack}}*)(TCGv_ptr cpu_env, TCGv next_eip)
        \end{lstlisting}
        The pop operation contains serveral steps.
        \begin{enumerate}
            \item Check if the stack is empty. \\
                If \verb|shack_top| equals to \verb|shack|, the statck is empty and no other operations need to be performed.

            \item Check if the guest address of the top entry matches \verb|next_eip|.  \\
                If the \verb|shadow_pair| on top of the shack contains the \verb|guest_eip| same as \verb|next_eip|,
                go to the next step; else no other operations need to be performed.

            \item Check if the host address of the top entry is valid. \\
                If the \verb|shadow_pair| contains a valid host address (not \verb|NULL|),
                go to the next step; else no other operations need to be performed.

            \item Update return address. \\
                Update the return address by modifying \verb|gen_opc_ptr| as stated in homework guildlines.
        \end{enumerate}


\section{Indirect Branch Target Cache}
    \subsection{Data Structure}
        \begin{lstlisting}[style=Structure]
            //optimization.h
            struct jmp_pair{
                target_ulong guest_eip;
                TranslationBlock *tb;
            };

            struct ibtc_table{
                struct jmp_pair htable[IBTC_CACHE_SIZE];
            };
        \end{lstlisting}
        IBTC is implemented using a direct map.
        Unlike the hash list used in shadow shack, each entry can only hold a single value instead of a chain.
        When a hash collision occurs, only the new value will be saved.
        \\
        \begin{lstlisting}[style=Structure]
            //optimization.c
            __thread int update_ibtc;
            struct ibtc_table *ibtc;
            target_ulong saved_eip;
        \end{lstlisting}
        \verb|update_ibtc| is a flag indicating whether \verb|update_ibtc_entry()| should be called in \verb|cpu-exec.c|.
        Since \verb|helper_lookup_ibtc()| is always executed before \verb|update_ibtc_entry()|,
        I use \verb|saved_eip| to preserve the guest address recieved in function \verb|helper_lookup_ibtc()|.

    \newpage

    \subsection{Functions}
        % ibtc_init
        \begin{lstlisting}[style=Function]
            void (*\textbf{\textcolor{blue}{ibtc\_init}}*)(CPUState *env)
        \end{lstlisting}
        In this funcion, \verb|ibtc_table| is initialized.
        \\\\
        % helper_lookup_ibtc
        \begin{lstlisting}[style=Function]
            void *(*\textbf{\textcolor{blue}{helper\_lookup\_ibtc}}*)(target_ulong guest_eip)
        \end{lstlisting}
        When there is an indirect jump instruction, this function helps to look up IBTC.
        If cache hits, return the saved host address.
        Otherwise, set flag \verb|update_ibtc| to $1$, set \verb|saved_eip| = \verb|guest_eip| and return \verb|optimization_ret_addr|.

        Since \verb|ibtc_table| holds $2^{16}$ entries, the $16$ least significant bits of \verb|guest_eip| are used as hash key.
        \\\\
        % update_ibtc_entry
        \begin{lstlisting}[style=Function]
            void (*\textbf{\textcolor{blue}{update\_ibtc\_entry}}*)(TranslationBlock *tb)
        \end{lstlisting}
        This function will be called if flag \verb|update_ibtc| is set. 
        It will insert an entry holding \verb|saved_eip| and the corresponding \verb|translationBlock| into \verb|ibtc_table|. 
        If a collision occurs, only the newly created entry will be preserved.


\section{Experiment}
    I ran two benchmarks MiBench and CoreMark to measure the correctness and effectiveness of my implementation.
    \subsection{Environment}
        \begin{itemize}
            \item Virtualbox Ubuntu 32-bit
            \item 1 core, Intel(R) Xeon(R) CPU E3-1230 V2 @ 3.30GHz
            \item 2G memory
        \end{itemize}

    \subsection{MiBench}
        % Without optimization, ibtc, shadow stack, both
        % basicmath, ..., ..., susan
        % time, speedup

    \subsection{MiBench}
        % Without optimization, ibtc, shadow stack, both
        % basicmath, ..., ..., susan
    

 
\end{document}
